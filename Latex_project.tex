\documentclass[12pt]{article}
\usepackage{amsmath}
\usepackage{graphicx}
\usepackage{hyperref}
\usepackage[latin1]{inputenc}

\title{Wykresy funkcji}
\author{Jan Krawczyk}
\date{2 B}

\begin{document}
\maketitle
$$ f(x) = x^3 + 4 x^2 + 4 x $$
 1) Dziedzina  funkcji 
$$ x \in \mathbb{R}$$
2) Miejsca zerowe funkcji
$$ x^3 + 4x^2 + 4x = 0$$
$$ x = 0 \vee x^2 + 4x + 4 = 0$$
$$ x = 0 \vee x = -2$$
3) Miejsce przeciecia z osia OY
$$ x = 0 \Leftrightarrow f(x) = y$$
$$ y = 0$$
4) Parzystosc lub nieparzystosc funkcji \\
Funkcja nie jest ani parzysta, ani nieparzysta\\
5) Granice
$$ lim _{x\to\infty} x^3+4x^2+4x = x^3(1+\frac4x+\frac4{x^2}) = \infty$$
$$ lim _{x\to-\infty} x^3+4x^2+4x = x^3(1+\frac4x+\frac4{x^2}) = -\infty$$
6)Asymptoty:\\
a) Pionowe: brak \\
b) Poziome: brak \\
c) Ukosne: brak \\
7) Monotonicznosc
$$ f\prime(x) = 3x^2+8x+4$$
$$ 3x^2+8x+4 = 0$$
$$ \Delta = 64-48$$
$$ \sqrt \Delta = 4$$
$$ x = -\frac2 3 \vee x = -2$$
$$ f\prime(x) > 0, x \in (-\infty;-2) \cup (-\frac2 3; \infty)$$
$$ f\prime(x) < 0, x \in (-2; -\frac2 3)$$
$$ f \nearrow (-\infty; -2\rangle , \langle -\frac2 3; \infty) $$
$$ f \searrow  \langle -2; -\frac2 3 \rangle $$
8) Ekstremum \\
funkcja osiaga max w:
$$ f(-2) = 0$$
funkcja osiaga min w:
$$ f(- \frac2 3) = -1 \frac5 {27}$$
9) Tabela\\
\begin{center}
\begin{tabular}{ c | c | c | c | c | c | c | c}
 $x$ & $(-\infty;-2)$ & $(-2)$ & $(-2;-\frac2 3)$ & $(-\frac2 3)$ & $(-\frac2 3;0)$ & $(0)$ & $(0; \infty)$\\
 \hline
 $f\prime(x)$ & $+$ & $0$ & $-$ & $0$ & $+$ & $+$ &$+$\\
 \hline
 $f(x)$ & $_{-\infty}\nearrow$ & $0^{max}$ & $\searrow$ & $-1\frac5 {27}^{min}$ & $\nearrow$ & $0$ & $\nearrow$\\    
\end{tabular}
\end{center}
10) Wykres\\

$$ f(x) = \frac {x^2-16}{x^2-4} $$
 1) Dziedzina  funkcji 
$$ x \in \mathbb{R} \backslash \{ -2;2 \}$$
2) Miejsca zerowe funkcji
$$ \frac {x^2-16}{x^2-4} = 0$$
$$ x^2 - 16 = 0$$
$$ x = -4 \vee x = 4$$
3) Miejsce przeciecia z osia OY
$$ x = 0 \Leftrightarrow f(x) = y$$
$$ y = \frac{-16}{-4} = 4$$
4) Parzystosc lub nieparzystosc funkcji \\
$$ f(-x) = \frac{(-x)^2-16}{(-x)^2-4} = \frac {x^2-16}{x^2-4} = f(x)$$ 
Funkcja jest parzysta\\
5) Granice
$$ lim _{x\to\infty} \frac {x^2-16}{x^2-4} = \frac{x^2(1-\frac{16}{x^2})}{x^2(1-\frac{4}{x^2})} = 1$$
$$ lim _{x\to-\infty} \frac {x^2-16}{x^2-4} = \frac{x^2(1-\frac{16}{x^2})}{x^2(1-\frac{4}{x^2})} = 1$$
$$ lim _{x\to2^-}$$
6)Asymptoty:\\
a) Pionowe: $ x=-2 \wedge x =2$\\
b) Poziome: $y = 1$\\
c) Ukosne: brak \\
7) Monotonicznosc
$$ f\prime(x) = \frac{2x(x^2-4)-2x(x^2-16)}{(x^2-4)^2} = \frac{2x^3-8x-2x^3+32x}{(x^2-4)^2} = \frac{24x}{(x^2-4)^2}$$
$$ \frac {24x}{(x^2-4)^2}= 0$$
Znak pochodnej zalezy od licznika
$$ 24x = 0$$
$$ x = 0$$
$$ f\prime(x) > 0, x \in (0; \infty)$$
$$ f\prime(x) < 0, x \in (-\infty; 0)$$
$$ f \nearrow \langle 0; \infty) $$
$$ f \searrow  ( -\infty; 0 \rangle $$
8) Ekstremum \\
funkcja osiaga max w:
$$ f(0) = 4$$
9) Tabela\\
\begin{center}
\begin{tabular}{ c | c | c | c | c | c | c | c}
 $x$ & $(-\infty;-2)$ & $(-2)$ & $(-2;0)$ & $(0)$ & $(0;2)$ & $(2)$ & $(2; \infty)$\\
 \hline
 $f\prime(x)$ & $-$ & $-$ & $-$ & $0$ & $+$ & $+$ &$+$\\
 \hline
 $f(x)$ & $^1\searrow_{-\infty}$ & $0$ & $^{\infty}\searrow_{-4}$ & $-4^{min}$ & $_{-4}\nearrow^{+\infty}$ & $0$ & $_{-\infty}\nearrow^1$\\    
\end{tabular}
\end{center}
10) Wykres\\
\end{document}
